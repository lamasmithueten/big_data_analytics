% !TEX root =  master.tex
\chapter{Säubern der Daten}

Nun ist die CSV vorbereitet für die eigentliche Säuberung. Die Daten liegen in den entsprechenden Spalten vor und alle uninteressanten Daten wurden entfernt.
Bevor die Daten jedoch in einer Datenbank in Tabellenform gespeichert werden können, sollten noch entsprechende Spalten bereinigt werden, dass nur noch Zahlenwerte ohne Einheiten dort aufzufinden sind.
Weiterhin beinhalten viele Spaltennamen immer noch Sonderzeichen und Leerzeichen, welche es schwierig macht die Spalten zu lesen.

Um diese Probleme zu addressieren, wurde R zur Bereinigung der Daten ausgewählt.
R ist eine Skriptsprache für statistische Berechnungen \cite{noauthor_r_nodate}. Es bietet mächtige Werkzeuge zur Auswertung und Speicherung von Daten \cite{noauthor_r_nodate}. Damit ist damit eine exzellente Wahl um dieses Problem zu lösen.
Es bietet Möglichkeiten die Spaltennamen automatisch mit der Funktion  \lstinline |clean_names()| aus der Bibliothek \lstinline|janitor| umzubenennen \cite{noauthor_janitor_nodate}, indem es problematische Zeichen, wie z.B. Leerzeichen und Sonderzeichen, erkennt und durch unproblematische Zeichen ersetzt.
Weiterhin werden auch alle Großbuchstaben durch Kleinbuchstaben ersetzt.
Darüber hinaus ist R eine dynamisch typisierte Skriptsprache. Datentypen werden je nach Kontext automatisiert erkannt.
Wenn die entsprechenden Spalten von allen nicht-numerischen Zeichen gesäubert werden, lassen sich diese problemlos in Zahlen umwandeln.
Die entsprechenden Einheiten werden nach der Spaltensäuberung stattdessen noch in den Spaltennamen hinzugefügt.

Wenn alle Spalten und Spaltennamen gesäubert bzw. angepasst wurden, können die Daten in einer Datenbank gespeichert werden.

Mit den vorher genannten Überlegungen im Hinterkopf ist folgendes Skript entstanden: 


\lstset{
	breaklines=true,         % Enable line wrapping
	breakatwhitespace=false, % Allow breaks at any character (not just whitespace)
	basicstyle=\ttfamily,    % Use monospaced font
}

\begin{lstlisting}[caption={\texttt{database.R}},captionpos=b]
#!/usr/bin/Rscript

library(RPostgres)

intel <- read.csv("clean.csv")		#Reading
intel <- janitor::clean_names(intel)	#Cleaning the column names

#cleaning and mutating columns to numbers
intel$max_turbo_taktfrequenz <- as.numeric(gsub(" GHz", "", intel$max_turbo_taktfrequenz))
intel$lithographie <- gsub("Intel 7", "10 nm", intel$lithographie)
intel$lithographie <- as.numeric(gsub(" nm", "", intel$lithographie))
intel$intel_turbo_boost_technik_2_0_taktfrequenz <- as.numeric(gsub(" GHz", "", intel$intel_turbo_boost_technik_2_0_taktfrequenz))
intel$grundtaktfrequenz_des_prozessors <- gsub(" \\| ", ".", intel$grundtaktfrequenz_des_prozessors)
intel$grundtaktfrequenz_des_prozessors <- as.numeric(gsub(" GHz", "", intel$grundtaktfrequenz_des_prozessors))
intel$cache <- gsub(" MB Intel Smart Cache", "", intel$cache)
intel$cache <- as.numeric(gsub(" MB", "", intel$cache))
intel$bus_taktfrequenz <- as.numeric(gsub(" GT/s", "", intel$bus_taktfrequenz))
intel$verlustleistung_tdp <- as.numeric(gsub(" W", "", intel$verlustleistung_tdp))
intel$intel_turbo_boost_max_technology_3_0_frequency <- gsub(" \\| ", ".", intel$intel_turbo_boost_max_technology_3_0_frequency)
intel$intel_turbo_boost_max_technology_3_0_frequency <- as.numeric(gsub(" GHz", "", intel$intel_turbo_boost_max_technology_3_0_frequency))
intel$single_p_core_turbo_frequency <- as.numeric(gsub(" GHz", "", intel$single_p_core_turbo_frequency))
intel$single_e_core_turbo_frequency <- as.numeric(gsub(" GHz", "", intel$single_e_core_turbo_frequency))
intel$e_core_base_frequency <- gsub(" GHz", "", intel$e_core_base_frequency)
intel$e_core_base_frequency <- as.numeric(gsub("900 MHz", "0.9", intel$e_core_base_frequency))
intel$total_l2_cache <- as.numeric(gsub(" MB", "", intel$total_l2_cache))
intel$processor_base_power <- as.numeric(gsub(" W", "", intel$processor_base_power))
intel$maximum_turbo_power <- as.numeric(gsub(" W", "", intel$maximum_turbo_power))
intel$grundtaktfrequenz_der_grafik <- as.numeric(gsub(" MHz", "", intel$grundtaktfrequenz_der_grafik))
intel$max_dynamische_grafikfrequenz <- as.numeric(gsub(" GHz", "", intel$max_dynamische_grafikfrequenz))
intel$max_videospeicher_der_grafik <- as.numeric(gsub(" GB", "", intel$max_videospeicher_der_grafik))
intel$x4k_unterstutzung <- gsub("Hz", "", intel$x4k_unterstutzung)
intel$x4k_unterstutzung <- as.numeric(gsub("Yes \\|  at ", "", intel$x4k_unterstutzung))


#Putting the units back into the column names
colnames(intel)[colnames(intel) == "max_turbo_taktfrequenz"] <- "max_turbo_taktfrequenz_GHz"
colnames(intel)[colnames(intel) == "lithographie"] <- "litographie_nm"
colnames(intel)[colnames(intel) == "intel_turbo_boost_technik_2_0_taktfrequenz"] <- "intel_turbo_boost_technik_2_0_taktfrequenz_GHz"
colnames(intel)[colnames(intel) == "grundtaktfrequenz_des_prozessors"] <- "grundtaktfrequenz_des_prozessors_GHz"
colnames(intel)[colnames(intel) == "cache"] <- "cache_MB"
colnames(intel)[colnames(intel) == "bus_taktfrequenz"] <- "bus_taktfrequenz_GT_per_s"
colnames(intel)[colnames(intel) == "verlustleistung_tdp"] <- "verlustleistung_tdp_W"
colnames(intel)[colnames(intel) == "intel_turbo_boost_max_technology_3_0_frequency"] <- "intel_turbo_boost_max_technology_3_0_frequency_GHz"
colnames(intel)[colnames(intel) == "single_p_core_turbo_frequency"] <- "single_p_core_turbo_frequency_GHz"
colnames(intel)[colnames(intel) == "single_e_core_turbo_frequency"] <- "single_e_core_turbo_frequency_GHz"
colnames(intel)[colnames(intel) == "e_core_base_frequency"] <- "e_core_base_frequency_GHz"
colnames(intel)[colnames(intel) == "total_l2_cache"] <- "total_l2_cache_MB"
colnames(intel)[colnames(intel) == "processor_base_power"] <- "processor_base_power_W"
colnames(intel)[colnames(intel) == "maximum_turbo_power"] <- "maximum_turbo_power_W"
colnames(intel)[colnames(intel) == "grundtaktfrequenz_der_grafik"] <- "grundtaktfrequenz_der_grafik_MHz"
colnames(intel)[colnames(intel) == "max_dynamische_grafikfrequenz"] <- "max_dynamische_grafikfrequenz_GHz"
colnames(intel)[colnames(intel) == "max_videospeicher_der_grafik"] <- "max_videospeicher_der_grafik_GB"
colnames(intel)[colnames(intel) == "x4k_unterstutzung"] <- "x4k_unterstutzung_at"


con <- dbConnect(
RPostgres::Postgres(),
dbname = "bda",
host = "localhost" ,
port = 5432,
user = "bda",
password = "bda",
)

dbWriteTable(con, "intel", intel, row.names = FALSE, overwrite = TRUE)
dbDisconnect(con)
\end{lstlisting}

Als erstes wird die transponierte CSV eingelesen und die Spaltennamen mit Hilfe der `janitor` Library gesäubert. Danach kommt ein großer Block an Funktionen, die die einzelnen Spalten von Einheiten und komischer Formatierung säubern.
Danach kommt ein großer Block zur Umbenennung der Spaltennamen. 
Im letzten Schritt wird der Dataframe in einer existierenden Datenbank gespeichert. 
Ein großer Vorteil an der Funktion ist, dass die Tabelle innerhalb der Datenbank vorher nicht existieren muss. 
R erstellt von selber eine neue Tabelle mit entsprechenden Datentypen basierend auf den Datentypen innerhalb des Dataframes. Daher ist es wichtig die Daten innerhalb des Skripts schon zu Nummern zu konvertieren.
