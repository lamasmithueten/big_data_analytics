% !TEX root =  master.tex
\chapter{Visualisieren der Daten (Phil Richter)}
Nach der Bereinigung und dem Speichern der Daten in einer Datenbank, wie in den vorherigen Kapiteln beschrieben, können die Daten visualisiert werden. Dafür wird als Grundlage
die Programmiersprache Python verwendet. Diese ist vorallem im Bereich der Datenanalyse und -visualisierung sehr weit verbreitet.

Für das Laden und Visualisiern der Daten werden folgende Packages verwendet:
\begin{table}[h!]
    \centering
    \begin{tabularx}{\textwidth}{|c|c|>{\centering\arraybackslash}X|}
        \hline
        \textbf{Package} & \textbf{Version} & \textbf{Beschreibung} \\ \hline
        dotenv & 1.0.1 & Lädt Umgebungsvariablen aus einer \textit{.env} Datei \\ \hline
        sqlalchemy & 2.0.36 & Ermöglicht Zugriff auf die Datenbank, sowie Datenbankabfragen \\ \hline
        pandas & 2.2.3 & Ermöglicht die Manipulation, Analyse und Verarbeitung von Daten \\ \hline
        matplotlib & 3.9.2 & Erstellt aus gegebenen Daten anpassbare Diagramme \\ \hline
        seaborn & 0.13.2 & Basiert auf \textit{matplotlib} und wird ebenfalls zur Datenvisualisierung verwendet \\ \hline
    \end{tabularx}
    \caption{Auflistung aller verwendeten Packages, sowie ihrer Versionen}
\end{table}

\section{Laden der Daten}
Im ersten Schritt der Visualisieren